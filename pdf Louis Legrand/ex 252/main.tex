\documentclass{article}
\usepackage{amsfonts,amsmath}

\title{Dévellopement de taylor}
\author{Romain Lemahieu}

\newcommand*{\R}{\mathbb{R}}
\newcommand*{\N}{\mathbb{N}}
\newcommand*{\p}[1]{$\mathcal{P}_{#1}$}
\begin{document}
\maketitle
\section{Enoncé}
Soit $f$ une fonction de $\R$ dans $\R$
admettant des dérivées de tout ordre
Montrer que pour tout $n$ dans $\N$ on a :

$$
  \forall x \in \R, 
  f(x)=\sum_{k=0}^{n}\frac{f^{(k)}(0)}{k!}x^k
  +
  \int_{0}^{x}
  \frac{(x-t)^n}{n!}
  f^{(n+1)}(t)\,\,dt 
$$

\section{Réponse}
Procédons à un raisonnement par récurence
Soit \p{n} : 
$$
f(x)=\sum_{k=0}^{n} \frac{f^{(k)}(0)}{k!}x^k
+
\int_{0}^{x}
\frac{(x-t)^{n}}{n!}
f^{(n+1)}(t)\,dt 
$$

\section{Initialisation : montrons que \p{0} est vrai}
\begin{align*}
\mathcal{P}_{0} : I&=\sum_{k=0}^{0} \frac{f^{(k)}(0)}{k!}x^k
+
\lim_{n \to 0}
\int_{0}^{x}
\frac{(x-t)^{n}}{0!}
f^{(0+1)}(t)\,dt\\
I&=f(0)+
\int_{0}^{x}
\lim_{n \to 0}
(x-t)^{n}
f'(t)\,dt\\
I&=f(0)+\Big[f(t)\Big]_{0}^{x} \textrm{ car} \lim_{x \to 0}x^0=1\\
I&=f(0)+f(x)-f(0)\\
I&=f(x)
\end{align*}
\pagebreak
\section{Hérédité}
En supposant que \p{n} est vrai montrons que \p{n+1} est vrai :
\begin{align*}
\mathcal{P}_{n+1} : &f(x)=\sum_{k=0}^{n+1} \frac{f^{(k)}(0)}{k!}x^k+\int_{0}^{x}
\frac{(x-t)^{n+1}}{(n+1)!}
f^{(n+2)}(t)\,dt\\
\mathcal{P}_{n} :  &f(x)=\sum_{k=0}^{n} \frac{f^{(k)}(0)}{k!}x^k
+
\int_{0}^{x}
\frac{(x-t)^{n}}{n!}
f^{(n+1)}(t)\,dt\\
&f(x)=\sum_{k=0}^{n+1} \frac{f^{(k)}(0)}{k!}x^k
-
\frac{f^{(n+1)}(0)}{(n+1)!}x^{k+1}
+
\int_{0}^{x}
\frac{(x-t)^{n}}{n!}
f^{(n+1)}(t)\,dt
\end{align*}

Procédons à l'intégration par partie :
\begin{align*}
&I=\int_{0}^{x}
\frac{(x-t)^{n+1}}{(n+1)!}
f^{(n+2)}(t)\,dt
&I=\int_{0}^{x}u(t)v'(t)\,dt
\end{align*}
avec
\begin{align*}
  v'(t)&=f^{(n+2)}(t)&u(t)=\frac{(x-t)^{n+1}}{(n+1)!}\\
  v(t)&=f^{(n+1)}(t)&u'(t)=-\,\frac{(n+1)(x-t)^{n}}{(n+1)!}\\
  & &u'(t)=-\,\frac{(x-t)^{n}}{n!}\\
\end{align*}
\begin{align*}
&I=\big[v(t)u(t)\big]_{0}^{x}-\int_{0}^{x}v(t)u'(t)\,dt\\
&I=\Biggl[f(t)^{(n+1)}\frac{(x-t)^{n+1}}{(n+1)!}\Biggl]^{x}_{0}
-\int_{0}^{x}-\,\frac{(x-t)^{n}}{n!}f(t)^{(n+1)}\,dt\\
&I=f(x)^{(n+1)}\frac{(x-x)^{n+1}}{(n+1)!}
-f(0)^{(n+1)}\frac{(x-0)^{n+1}}{(n+1)!}
-\int_{0}^{x}-\,\frac{(x-t)^{n}}{n!}f(t)^{(n+1)}\,dt\\
&I=-f(0)^{(n+1)}\frac{x^{n+1}}{(n+1)!}
+\int_{0}^{x}\frac{(x-t)^{n}}{n!}f(t)^{(n+1)}\,dt\\
&I=-\,\frac{f(0)^{(n+1)}}{(n+1)!}x^{n+1}
+\int_{0}^{x}\frac{(x-t)^{n}}{n!}f(t)^{(n+1)}\,dt\\\\
&f(x)=\sum_{k=0}^{n+1} \frac{f^{(k)}(0)}{k!}x^k+I\\
&f(x)=\sum_{k=0}^{n+1} \frac{f^{(k)}(0)}{k!}x^k+\int_{0}^{x}
\frac{(x-t)^{n+1}}{(n+1)!}
f^{(n+2)}(t)\,dt
\end{align*}
donc \p{n+1} est verifié.

\section{Conclusion}
La propriété est vraie au rang initial \p{0} est héréditaire
donc d'après le principe de récurence
 donc elle est vraie pour tout entier naturel n.
\end{document}