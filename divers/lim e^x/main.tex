\documentclass{exam}
\usepackage{amsfonts,amsmath,graphicx}
\newcommand*{\p}{\mathcal{P}}
\newcommand*{\QS}{\mathbb{Q}^*}
\setlength{\parskip}{6pt}

\title{Limite de $e$}
\author{Romain Lemahieu}
\begin{document}
\maketitle
Montrer que :
$$
\lim_{x \to \infty} \left(1+\frac{1}{x}\right)^x=e
$$
\begin{align*}
f(x)&=\left(1+\frac{1}{x}\right)^x
\\
\ln(f(x))&=\ln\left(\left(1+\frac{1}{x}\right)^x\right)
\\
\ln(f(x))&=x\ln\left(\frac{x+1}{x}\right)
\\
\ln(f(x))&=x(\ln(x+1)-\ln(x))
\\
\ln(f(x))&=x\int_{x}^{x+1}\frac{1}{t}\,dt
\\
\text{Soit $t \in [x; x+1]$}
\\
\frac{1}{x+1}
&\le
\frac{1}{t}
\le
\frac{1}{x}
\\
(x+1-x)\frac{1}{x+1}
&\le
\int_{x}^{x+1}\frac{1}{t}\,dt
\le
(x+1-x)\frac{1}{x}
\\
\frac{1}{x+1}
&\le
\int_{x}^{x+1}\frac{1}{t}\,dt
\le
\frac{1}{x}
\\
\frac{x}{x+1}
&\le
\ln(f(x))
\le
\frac{x}{x}
\\
1-\frac{1}{x+1}
&\le
\ln(f(x))
\le
1
\\
\lim_{x \to +\infty}\frac{1}{1+x}=0
&\text{ par soustraction}
\lim_{x \to +\infty}\left(1-\frac{1}{1+x}\right)=1
\\
\text{donc d'après le théorème des gendarmes}
\quad
&\lim_{x \to +\infty} \ln(f(x))=1
\\
\text{or pour tout $x$ réel posisif}
\quad
e^{\ln(f(x))}&=f(x)
\\
\lim_{x \to 1} e^x=e
\quad
\text{donc par composition}
&\quad
\lim_{x \to +\infty} f(x)=e
\end{align*}
\end{document}