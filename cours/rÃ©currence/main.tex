\documentclass{exam}
\usepackage{amsfonts,amsmath,graphicx}
\newcommand*{\p}{\mathcal{P}}
\newcommand*{\n}{\mathbb{N}}
\newcommand*{\sys}[1]{
$$
\begin{cases}
#1
\end{cases}
$$
}

\setlength{\parskip}{6pt}
\title{Récurrence}
\author{Romain Lemahieu}
\begin{document}
\maketitle
\section{Première récurrence}
Soit $(u_n)_{n \in \n}$ 
\sys{
  u_0&=0\\
  u_{n+1}&=2u_n+1
}
Montrer que
$$
\forall n\in\n,
\quad
u_n=2^n-1
$$
\subsection{Définition}
Une propriété est une ``fonction" qui est soit vraie ou fausse, par exemple :
\begin{align*}
\p_1(x):&``x+1=4"
&&
\p_1(2)
\text{ est fausse}
&&
\p_1(3)
\text{ est vraie}
\\
\p_2(x):&``x<2"
&&
\p_2(2)
\text{ est fausse}
&&
\p_1\left(-\sqrt{3}\right)
\text{ est vraie}
\\
\p_3(x;y):&``x>y"
&&
\p_3(2;4)
\text{ est fausse}
&&
\p_3(-2;t^2)
\text{ est vraie pour tout réel }t
\end{align*}
Dans le raisonnement par récurrence classique utilise une propriété définie
sur un sous ensemble des entier naturel avec qu'un paramètre
\\
Rédaction : 
\\
Soit la propriété définie pour tout entier naturel
$\p(n):``u_n=2^n-1"$
$$
\forall n\in\n,
u_n=2^n-1
\Leftrightarrow
\p(n) \text{ est vraie}
$$
\subsection{Initialisation}
Ne pas faire :
$$
u_0=2^0-1=0
$$
\\
Rédaction :
\\
Montrons que $\p(0)$ est vraie :
$$
u_0=0 \quad 2^0-1=0
$$
Donc $\p$ est initialisée.
\subsection{Hérédité}
En supposant pour un certain entier naturel $k$ que $\p(k)$ est vraie
montrons que $\p(k+1)$ l'est :
\\
Broullion :
\\
Hypothèse de récurrence : $\p(k):``u_k=2^k-1"$
\\
Ce que on veut démontrer : $\p(k+1):``u_{k+1}=2^{k+1}-1"$

\subsubsection{En partant d'énoncé :}
\begin{align*}
u_{k+1}=&2u_k+1
&
\text{d'après l'énoncé}
\\
u_{k+1}=&2\times(2^k-1)+1
&
\text{par hypothèse de récurrence}
\\
u_{k+1}=&2\times2^k-2+1
\\
u_{k+1}=&2^{k+1}-1
\end{align*}
Donc $\p$ est héréditaire.
\subsubsection{En partant de la propriété :}
\begin{align*}
u_k=&2^k-1
&
\text{par hypothèse de récurrence}
\\
2u_k=&2\times\left(2^k-1\right)
\\
u_{k+1}=&2\times2^k-2+1
&
\text{d'après l'énoncé}
\\
u_{k+1}=&2^{k+1}-1
\end{align*}
Donc $\p$ est héréditaire
\subsection{Conclusion}
La propriété $\p$ étant initialisée au rang 0 et héréditaire $\p(n)$ est vraie pour
tout entier naturel $n$ donc :
$$
\forall n\in\n,
\quad
u_n=2^n-1
$$
\section{Exercice}
\begin{questions}
\question Soit $u_n$ definie pour tout entier naturel $n$ tel que :
\sys{
u_0&=-2\\
u_{n+1}&=-3u_n+2
}
Montrer que $
\forall n\in\n,
\quad
u_n=-2,5\times(-3)^n+0,5
$
\question Montrer que la somme des angles d'un polygone de $n$ côtés est égale à $180+180n$
\question Montrer l'inégalité de bernoulli : 
$$
(x+1)^n>1+nx
$$
pour tout entier naturel non nul et pour tout réel x non nul supérieur ou égal à -1
\question Montrer pour tout entier $n$ et tout réel $q$ différent de $1$ que :
$$
\sum_{k=0}^{n}q^n=\frac{1-q^{n+1}}{1-q}
$$
\question (Récurrence Double) Exercice 7 LLG. Soit $u_n$ definie pour tout entier naturel $n$ tel que :
\sys{
u_0 = 2
\\
u_1 = 5
\\
u_{n+2} = 5u_{n+1} - 6u_n.
}
Montrer que
$\forall n\in\n \quad u_n=2^n+3^n$
\question (Récurrence forte) Montrer que pour tout entier naturel $n$ strictement supérieur à 1 
s'exprime comme un produit de facteurs premiers d'une unique façon, à l'ordre près des facteurs.
\question Polynômes de Tchebychev de première espèce (Ex CGL 2023).
Soit $T_n$ definie pour tout entier naturel $n$ tel que :
\sys{
T_0(x)&=0
\\
T_1(x)&=x
\\
T_{n+2}&=2xT_{n+1}(x)-T_n(x)
}
Montrer que $T_n(\cos(\theta))=\cos(n\theta)$, sachant que:
\begin{align*}
\cos(a+b)=\cos(a)cos(b)-\sin(a)\sin(b)\\
\cos(a-b)=\cos(a)cos(b)+\sin(a)\sin(b)
\end{align*}
\question soit 
\sys{
a_0=1\\
b_0=0\\
c_0=0\\
}
et
\sys{
a_{n+1}=\dfrac{1}{3}a_n+\dfrac{1}{4}b_n\\
\\
b_{n+1}=\dfrac{2}{3}a_n+\dfrac{1}{2}b_n+\dfrac{2}{3}c_n\\
\\
c_{n+1}=\dfrac{1}{4}b_n+\dfrac{1}{3}c_n\\
}
montrer que $\forall n\in\n$ :
\sys{
a_n=\dfrac{3}{14}+\dfrac{2}{7}\left(\dfrac{1}{3}\right)^n+\dfrac{2}{7}\left(-\dfrac{1}{6}\right)^n\\
\\
b_n=\dfrac{4}{7}-\dfrac{4}{7}\left(-\dfrac{1}{6}\right)^n\\
\\
c_n=\dfrac{3}{14}-\dfrac{1}{2}\left(\dfrac{1}{3}\right)^n+\dfrac{2}{7}\left(-\dfrac{1}{6}\right)^n\\
}
\question Exercice 1 Concours général 2023
\end{questions}
\end{document}